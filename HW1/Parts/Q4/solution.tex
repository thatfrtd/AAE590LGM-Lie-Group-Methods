% !TeX root = ../../main.tex
\documentclass[../AAE590LGM_HW1.tex]{subfiles}
\begin{document}

\subsection{4.a)}
\underline{Question}: \\
Write the general form of an element $\Omega \in \soa{2}$ \\
\underline{Solution}: \\
$2\times 2$ skew symmetric matrices can be paramterized using $\theta \in \R$ as the following
\begin{gather*}
    \Omega = \begin{pmatrix}0 & -\theta \\ 
    \theta & 0\end{pmatrix} \in \soa{2}
\end{gather*}

\subsection{4.b)}
\underline{Question}: \\
Define the wedge operator $(\cdot)^\wedge:\R\to \soa{2}$ and its inverse, the vee operator $(\cdot)^\wedge:\soa{2}\to\R$ \\
\underline{Solution}: \\
For $SO{2}$ the operations are defined as the following: \\
The wedge operator is defined as $\theta^\wedge = \begin{pmatrix}0 & -\theta \\ \theta & 0\end{pmatrix} = \Omega$\\
The vee operator is defined as $\Omega^\vee = \Omega_{21} = \theta$

\subsection{4.c)}
\underline{Question}: \\
Compute the matrix exponential $\exp(\Omega)$ for $\Omega = \begin{pmatrix}0 & -\theta \\ \theta & 0\end{pmatrix}$ using the power series definition.
Show that this equals $R(\theta)$ \\
\underline{Solution}: \\
Define the matrix $J = \begin{pmatrix}0 & -1 \\ 1 & 0\end{pmatrix}$
\begin{gather*}
    \exp(\Omega) = I + \Omega + \frac{1}{2}\Omega^2 + \frac{1}{6}\Omega^3 + \frac{1}{24}\Omega^4 + \dots \\
    \Omega^2 = \begin{pmatrix}0 & -\theta \\ \theta & 0\end{pmatrix} \begin{pmatrix}0 & -\theta \\ \theta & 0\end{pmatrix} = -\theta^2I \\
    \implies \Omega^3 = -\theta^2\Omega \\ 
    \implies \exp(\Omega) = (1 - \frac{1}{2}\theta^2 + \frac{1}{24}\theta^4 + \dots)I + (\theta - \frac{1}{6}\theta^3 + \frac{1}{120}\theta^5 + \dots)J \\
    \exp(\Omega) = \cos(\theta)I + \sin(\theta)J = 
    \begin{pmatrix}
        \cos(\theta) & -\sin(\theta) \\ 
        \sin(\theta) & \cos(\theta)
    \end{pmatrix} = R(\theta) \\
    \therefore \exp(\Omega) = R(\theta)
\end{gather*}

\subsection{4.d)}
\underline{Question}: \\
What is the dimension of $\soa{2}$? How does this relate to the dimension of the of $\SO{2}$? \\ 
\underline{Solution}: \\
The dimension of $\soa{2}$ is 1 because the elements, $2\times2$ skew-symmetric matrices, are parametrized by a single number $\theta$. 
The dimension of $\SO{2}$ is 1 because the 2 dimensional rotation matrices are also parametrized by a single number $\theta$.
The dimensions of the Lie group and Lie algebra are the same for $\SO{2}$ and $\soa{2}$. In fact, this holds for any Lie group.


\end{document}