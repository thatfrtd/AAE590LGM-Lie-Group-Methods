% !TeX root = ../../main.tex
\documentclass[../AAE590LGM_HW1.tex]{subfiles}
\begin{document}

\subsection{2.a)}
\underline{Question}: \\
The set of $2\times 2$ invertible matrices with real entries under matrix multiplication \\
\underline{Solution}: \\
\begin{itemize}
\item[Closure] 
All invertible matrices have nonzero determinants. $\text{det}(AB) = \text{det}A\text{det}(B)$. Therefore, the only way for 
the determinant of the product of two matrices to be zero is if one of the matrices has a determinant of zero. However, both 
matrices are invertible so they have nonzero determinants so their product is a $2\times2$ invertible matrix with real entries.
Thus, closure holds.
\item[Associativity]
Matrix multiplication is always associative so this property is inherited.
\item[Identity]
The identity matrix multiplied with any matrix is the matrix itself so the identity matrix is the identity for this set under matrix multiplication.
\item[Inverse]  
The set by definition contains only invertible matrices so every element can be inverted to find its multiplicative inverse.
\end{itemize}
Therefore, all group axioms are satisfied for this set-operation pair making it a group.

\subsection{2.b)}
\underline{Question}: \\
The set of integers $\Z$ under addition \\
\underline{Solution}: \\
\begin{itemize}
\item[Closure] 
Adding two integers yields another integer so they are closed under addition.
\item[Associativity]
Addition on integers is always associative so this property is satisfied.
\item[Identity]
The identity element under addition is 0.
\item[Inverse]  
The additive inverse element can be found by negating the element which is still an integer so inverses always exist.
\end{itemize}
Therefore, all group axioms are satisfied for this set-operation pair making it a group.

\subsection{2.c)}
\underline{Question}: \\
The set of ${1, -1, i, -i}$ under complex multiplication \\
\underline{Solution}: \\
\begin{itemize}
\item[Closure] 
\begin{gather*}
    1^2 = 1, \: (-1)^2 = 1, \: i^2 = -1, \: (-i)^2 = -1 \\
    1 \times -1 = -1, \: 1 \times i = i, \: 1 \times -i = -i \\
    -1 \times i = -i, \: -1 \times -i = i \\
    i \times -i = 1
\end{gather*}
All of the products between elements of the set are still in the set, therefore,
the set is closed under complex multiplication.
\item[Associativity]
Complex multiplication is always associative so the property is inherited.
\item[Identity]
Looking at the products above, it is clear that the identity element is $1$.
\item[Inverse]  
Looking at the products above, the inverse of $1$ is $1$, the inverse of $-1$ is $-1$, 
the inverse of $i$ is $-i$, and the inverse of $-i$ is $i$. Therefore, every element has an inverse.
\end{itemize}
This set and operation pair pass all of the axioms so it is a group.

\subsection{2.d)}
\underline{Question}: \\
The set of $2\times 2$ with determinant 1 under matrix addition \\
\underline{Solution}: \\
\begin{itemize}
\item[Closure] 
The identity matrix has a determinant of 1 so it is in the set. If the set is a group under matrix addition then twice the 
identity matrix should be in the set.
\begin{gather*}
    \text{det}(I) = 1 \\
    \text{det}(2I) = 2^2\text{det}(I) = 4 \neq 1
\end{gather*}
Therefore, the set is not closed under matrix addition so it is not group with this choice of operation.
\end{itemize}

\subsection{2.e)}
\underline{Question}: \\
The set of positive real numbers $\R[+]$  under multiplication \\
\underline{Solution}: \\
\begin{itemize}
\item[Closure] 
The product of any two positive numbers is positive so closure is satisfied.
\item[Associativity]
Real multiplication is always associative so the property is inherited.
\item[Identity]
The product of any real number with 1 is itself so the property is inherited.
\item[Inverse]  
The product of any real number with its reciprical (which is always a real number) is 1 so there always exists an inverse element.
\end{itemize}
Therefore, the set and operation pair passes all of the axioms so it is a group.

\end{document}