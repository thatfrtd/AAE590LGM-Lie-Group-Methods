% Problem Set 02: SE(2) - Planar Rigid Body Motions
\documentclass[11pt]{article}
\usepackage[margin=1in]{geometry}
\usepackage{amsmath,amssymb,amsthm}
\usepackage{enumitem}
\usepackage{hyperref}
\usepackage{bm}

% Custom commands
\newcommand{\so}{\mathfrak{so}}
\newcommand{\se}{\mathfrak{se}}
\newcommand{\tra}{\mathfrak{t}}
\newcommand{\SO}{\mathrm{SO}}
\newcommand{\SE}{\mathrm{SE}}
\newcommand{\T}{\mathrm{T}}
\newcommand{\Ad}{\mathrm{Ad}}
\newcommand{\ad}{\mathrm{ad}}
\newcommand{\Exp}{\mathrm{Exp}}
\newcommand{\Log}{\mathrm{Log}}
\newcommand{\R}{\mathbb{R}}
\newcommand{\Z}{\mathbb{Z}}

\title{AAE 590: Problem Set 02\\SE(2): Planar Rigid Body Motions}
\author{}
\date{Due: See Brightspace}

\begin{document}
\maketitle

\section*{Instructions}
\begin{itemize}
    \item Show all work for full credit. Derivations should be clear and complete.
    \item \textbf{Submit a single PDF to Gradescope} containing:
    \begin{itemize}
        \item Derivations (handwritten and scanned, or typed in \LaTeX)
        \item Python code (included as text or screenshots)
        \item All plots and numerical verification results
    \end{itemize}
    \item You may use AI tools, but you must understand your solutions (validated via in-class quizzes).
    \item \textbf{Prerequisite:} You should have working implementations of \texttt{so2\_wedge}, \texttt{so2\_vee}, \texttt{so2\_exp}, and \texttt{so2\_log} from PS01.
    \item Estimated time: 6 hours
\end{itemize}

%==============================================================================
\section*{Problem 1: Normal Subgroups and Quotient Groups (15 points)}
%==============================================================================

Before diving into $\SE(2)$, let's understand why normal subgroups matter.

\begin{enumerate}[label=(\alph*)]
    \item (5 pts) \textbf{Definition and Intuition:}

    A subgroup $N \leq G$ is \textbf{normal} (written $N \trianglelefteq G$) if $gNg^{-1} = N$ for all $g \in G$.
    \begin{itemize}
        \item Explain why every subgroup of an abelian group is automatically normal.
        \item For non-abelian groups, conjugation can ``twist'' a subgroup. Give geometric intuition: if $H$ is a set of translations and $g$ is a rotation, what does $gHg^{-1}$ represent?
    \end{itemize}

    \item (5 pts) \textbf{A Matrix Group Example:}

    Consider the determinant map $\det: \mathrm{GL}(2, \R) \to \R^*$ (where $\R^* = \R \setminus \{0\}$ under multiplication).
    \begin{itemize}
        \item Verify $\det$ is a homomorphism: $\det(AB) = \det(A) \det(B)$.
        \item What is $\ker(\det)$? (This group has a name---what is it?)
        \item What information about a matrix is ``forgotten'' when we apply $\det$? What's preserved?
    \end{itemize}

    \item (5 pts) \textbf{First Isomorphism Theorem (Preview):}

    For a homomorphism $\phi: G \to H$:
    \begin{itemize}
        \item \textbf{Kernel:} $\ker(\phi) = \{g \in G : \phi(g) = e_H\}$ (elements mapped to identity)
        \item \textbf{Image (Range):} $\mathrm{im}(\phi) = \{\phi(g) : g \in G\}$ (elements that $\phi$ ``hits'')
    \end{itemize}

    \textbf{First Isomorphism Theorem:} $G/\ker(\phi) \cong \mathrm{im}(\phi)$.

    \textit{In words:} quotienting by what $\phi$ ``kills'' gives you what $\phi$ ``sees.''
    \begin{itemize}
        \item For $\SE(2)$: if we define $\pi: \SE(2) \to \SO(2)$ by $\pi(\bm{t}, R) = R$, what is $\ker(\pi)$?
        \item What does the First Isomorphism Theorem tell us about $\SE(2)/\ker(\pi)$?
    \end{itemize}
\end{enumerate}

%==============================================================================
\section*{Problem 2: SE(2) and the Semi-Direct Product (35 points)}
%==============================================================================

The group $\SE(2)$ of planar rigid motions is our first example of a \textbf{semi-direct product}. This structure---where one subgroup ``twists'' another---is fundamental to robotics.

\medskip
\noindent\textbf{The Translation Group $\T(2)$:} The set of 2D translations forms a group $\T(2)$ under composition. As a matrix Lie group:
\[
\T(2) = \left\{ \begin{bmatrix} I & \bm{t} \\ \bm{0}^T & 1 \end{bmatrix} : \bm{t} \in \R^2 \right\} \cong (\R^2, +)
\]
This group is abelian (translations commute) and isomorphic to $\R^2$ with vector addition. Its Lie algebra is $\tra(2) \cong \R^2$.

\begin{enumerate}[label=(\alph*)]
    \item (5 pts) \textbf{Matrix Representation:} Write $X \in \SE(2)$ as a $3 \times 3$ matrix:
    \[
    X = \begin{bmatrix} R & \bm{t} \\ \bm{0}^T & 1 \end{bmatrix}, \quad R \in \SO(2), \; \bm{t} \in \R^2
    \]

    \textbf{Tuple notation:} We can also write $X = (\bm{t}, R)$ as a compact tuple. The correspondence is:
    \[
    (\bm{t}, R) \;\longleftrightarrow\; \begin{bmatrix} R & \bm{t} \\ \bm{0}^T & 1 \end{bmatrix}
    \]
    We order as $(\bm{t}, R)$ (translation first) to match the semi-direct product $\T(2) \rtimes \SO(2)$ convention. Both notations represent the same rigid motion: rotate by $R$, then translate by $\bm{t}$ (in the world frame).
    \begin{itemize}
        \item Derive the composition $X_1 X_2$ and inverse $X^{-1}$ formulas using block matrix multiplication. You don't need to expand $2 \times 2$ blocks---leave products like $R_1 R_2$ as is.
        \item Express your results in tuple form: $(\bm{t}_1, R_1) \cdot (\bm{t}_2, R_2) = (?, ?)$ and $(\bm{t}, R)^{-1} = (?, ?)$.
        \item Implement \texttt{se2\_compose(X1, X2)} and \texttt{se2\_inverse(X)}.
    \end{itemize}

    \textit{Check your work:} $(\bm{t}_1, R_1) \cdot (\bm{t}_2, R_2) = (\bm{t}_1 + R_1 \bm{t}_2, \; R_1 R_2)$ and $(\bm{t}, R)^{-1} = (-R^T \bm{t}, \; R^T)$.

    \item (5 pts) \textbf{Why SE(2) is Non-Abelian:}
    \begin{itemize}
        \item Compute $X_R X_t$ and $X_t X_R$ where $X_R = (\bm{0}, R_{90^\circ})$ and $X_t = ((1, 0)^T, I)$.
        \item Where does the origin end up under each? Sketch both results to see why order matters.
        \item Why does coupling rotations with translations break commutativity? (Hint: $\SO(2)$ alone is abelian.)
        \item \textbf{Action on points:} For $X = (\bm{t}, R)$, the action on a point is $X \cdot \bm{p} = R\bm{p} + \bm{t}$. Note that $(X_1 X_2) \cdot \bm{p} = X_1 \cdot (X_2 \cdot \bm{p})$ (the right factor acts first). Using your result for $X_R X_t$, find the world position of a sensor at body position $\bm{p}_b = (0.5, 0)^T$.
    \end{itemize}

    \item (10 pts) \textbf{The Semi-Direct Product Structure:}

    $\SE(2)$ is the semi-direct product $\SE(2) = \T(2) \rtimes \SO(2)$, where rotations \textit{act on} translations.

    For a general semi-direct product $N \rtimes_\phi H$, the group operation is:
    \[
    (n_1, h_1) \cdot (n_2, h_2) = (n_1 \cdot \phi(h_1)(n_2), \; h_1 h_2)
    \]
    where $\phi: H \to \mathrm{Aut}(N)$ describes how $H$ acts on $N$.

    \begin{itemize}
        \item For $\SE(2) = \T(2) \rtimes \SO(2)$: identify $N$, $H$, and the action $\phi$. What does $\phi(R)$ do to a translation $\bm{t}$?
        \item Apply the semi-direct product formula to show $(\bm{t}_1, R_1) \cdot (\bm{t}_2, R_2) = (\bm{t}_1 + R_1 \bm{t}_2, \; R_1 R_2)$. Verify this matches your composition formula from part (a).
        \item \textbf{Key insight:} In a direct product $N \times H$, neither factor affects the other. In a semi-direct product, $H$ ``twists'' $N$. Explain why ``translate then rotate'' $\neq$ ``rotate then translate.''
        \item Why is $\SE(2)$ written as $\T(2) \rtimes \SO(2)$ and not $\SO(2) \rtimes \T(2)$? (Hint: which subgroup is normal?)
    \end{itemize}

    \item (15 pts) \textbf{Normal Subgroups and the Quotient $\SE(2)/\T(2) \cong \SO(2)$:}

    The translation subgroup is $\T(2) = \{(\bm{t}, I) : \bm{t} \in \R^2\} \subset \SE(2)$, i.e., elements of the form:
    \[
    (\bm{t}, I) \;\longleftrightarrow\; \begin{bmatrix} I & \bm{t} \\ \bm{0}^T & 1 \end{bmatrix}
    \]
    \begin{itemize}
        \item Compute the conjugation $(\bm{p}, R) \cdot (\bm{t}, I) \cdot (\bm{p}, R)^{-1}$ using the tuple formula. Also verify by matrix multiplication. Is the result a pure translation?
        \item Based on your result, explain why $\T(2) \trianglelefteq \SE(2)$ (i.e., translations form a normal subgroup).
        \item Now try: is $\SO(2) = \{(\bm{0}, R)\}$ a normal subgroup? Compute $(\bm{t}, I) \cdot (\bm{0}, R) \cdot (\bm{t}, I)^{-1}$.
        \item Describe the cosets of $\T(2)$ in $\SE(2)$. What do two elements in the same coset share?
        \item Define $\pi: \SE(2) \to \SO(2)$ by $\pi(\bm{t}, R) = R$. Verify $\pi$ is a homomorphism with $\ker(\pi) = \T(2)$.
        \item Apply the First Isomorphism Theorem: $\SE(2)/\T(2) \cong \SO(2)$. Interpret: ``forgetting position leaves orientation.''

        \item \textbf{Predicting the Adjoint Structure (will verify in Problem 3):}

        Your conjugation results reveal the structure of $\Ad_X$ before you derive it!
        \begin{itemize}
            \item You showed $(\bm{p}, R) \cdot (\bm{t}, I) \cdot (\bm{p}, R)^{-1} = (R\bm{t}, I)$. The adjoint $\Ad_X$ is the linearization (derivative at the identity) of the conjugation map $Y \mapsto X Y X^{-1}$.
            \item \textbf{Predict:} Based on this, what $2 \times 2$ block should appear in $\Ad_X$ acting on the velocity components $(v_x, v_y)$? Explain your reasoning, and also predict whether translation affects angular velocity (Hint: is $\SO(2)$ normal?).
        \end{itemize}
        \textit{You'll verify your predictions in Problem 3(d).}
    \end{itemize}
\end{enumerate}

%==============================================================================
\section*{Problem 3: SE(2) Lie Algebra, Exp/Log, and Adjoint (35 points)}
%==============================================================================

The Lie algebra $\se(2)$ consists of twists. We use \textbf{translation-first} ordering $\xi = (v_x, v_y, \omega)^T \in \R^3$, which yields block upper-triangular matrices.

\medskip
\noindent\textbf{Notation:} $\Exp$ denotes the matrix exponential. It takes a matrix $\xi^\wedge \in \se(2)$ and returns a group element in $\SE(2)$.

\begin{enumerate}[label=(\alph*)]
    \item (5 pts) \textbf{Lie Algebra Structure:}

    With $\xi = (v_x, v_y, \omega)^T = (\bm{v}^T, \omega)^T$, the wedge map is:
    \[
    \xi^\wedge = \begin{bmatrix} \omega^\wedge & \bm{v} \\ \bm{0}^T & 0 \end{bmatrix} = \begin{bmatrix} 0 & -\omega & v_x \\ \omega & 0 & v_y \\ 0 & 0 & 0 \end{bmatrix}
    \]
    \begin{itemize}
        \item Verify this is block upper-triangular (rotation block in upper-left, translation in upper-right).
        \item Implement \texttt{se2\_wedge(xi)} and \texttt{se2\_vee(Xi)}.
    \end{itemize}

    \item (8 pts) \textbf{Exponential Map:} The closed-form expression is:
    \[
    \Exp(\xi^\wedge) = \begin{bmatrix} R(\omega) & V(\omega) \bm{v} \\ \bm{0}^T & 1 \end{bmatrix}
    \]
    where $V(\omega) = \frac{\sin\omega}{\omega} I + \frac{1 - \cos\omega}{\omega} \begin{bmatrix} 0 & -1 \\ 1 & 0 \end{bmatrix}$.
    \begin{itemize}
        \item Derive by hand using the matrix exponential power series (show key steps).
        \item Implement \texttt{se2\_exp(xi)} with Taylor series for $\omega \approx 0$ (use 5 terms).
    \end{itemize}

    \item (7 pts) \textbf{Logarithm Map:}

    The inverse of $V(\omega)$ is:
    \[
    V(\omega)^{-1} = \frac{\omega}{2} \cot\frac{\omega}{2} \, I + \frac{\omega}{2} \begin{bmatrix} 0 & 1 \\ -1 & 0 \end{bmatrix}
    \]
    \begin{itemize}
        \item Implement \texttt{se2\_log(X)}. Use Taylor series for $\omega \approx 0$ (use 5 terms).
        \item Verify numerically: $\Exp(\Log(X)) = X$ for random $X \in \SE(2)$.
    \end{itemize}

    \item (10 pts) \textbf{Adjoint Representation $\Ad_X$:}

    The adjoint $\Ad_X: \se(2) \to \se(2)$ is defined by $(\Ad_X \xi)^\wedge = X \xi^\wedge X^{-1}$.

    \textit{Note: You don't need to multiply out $2 \times 2$ blocks explicitly. Showing block form (e.g., $R \omega^\wedge R^T$, $R\bm{v}$) is sufficient.}

    \textbf{Translation-first ordering} $\xi = (v_x, v_y, \omega)^T$:
    \begin{itemize}
        \item Compute $X \xi^\wedge X^{-1}$ explicitly for $X = (\bm{t}, R)$.
        \item Extract the $3 \times 3$ matrix $\Ad_X$ such that $\Ad_X \xi$ gives the transformed twist.

        You should get: $\Ad_X = \begin{bmatrix} R & \bm{t}^\odot \\ \bm{0}^T & 1 \end{bmatrix}$ where $\bm{t}^\odot = \begin{bmatrix} t_y \\ -t_x \end{bmatrix} = -J\bm{t}$ with $J = \begin{bmatrix} 0 & -1 \\ 1 & 0 \end{bmatrix}$.
    \end{itemize}

    \textbf{Rotation-first ordering} $\xi = (\omega, v_x, v_y)^T$:
    \begin{itemize}
        \item Repeat the derivation with this ordering. The wedge map produces the same $3 \times 3$ matrix, but extracting $\Ad_X$ requires matching to the new vector ordering.
        \item You should get: $\Ad_X = \begin{bmatrix} 1 & \bm{0}^T \\ \bm{t}^\odot & R \end{bmatrix}$ (block \textit{lower}-triangular).
    \end{itemize}

    \textbf{Verification:}
    \begin{itemize}
        \item Verify numerically: $\Ad_{X_1 X_2} = \Ad_{X_1} \Ad_{X_2}$ for random $X_1, X_2 \in \SE(2)$.
        \item Verify numerically: $\Ad_X^{-1} = \Ad_{X^{-1}}$ (the adjoint respects inverses).
        \item \textbf{Connection to Problem 2:} Confirm that $R$ acts on velocity components, as you predicted from the normal subgroup structure.
    \end{itemize}

    \item (5 pts) \textbf{Lie Bracket and $\ad_\xi$:}

    The Lie bracket on vectors is defined by $[\xi_1, \xi_2]^\wedge := \xi_1^\wedge \xi_2^\wedge - \xi_2^\wedge \xi_1^\wedge$ (i.e., compute the matrix commutator, then apply vee). The small adjoint $\ad_\xi$ is the $3 \times 3$ matrix such that $[\xi_1, \xi_2] = \ad_{\xi_1} \xi_2$.

    \textbf{Translation-first ordering} $\xi = (v_x, v_y, \omega)^T$:
    \begin{itemize}
        \item Compute $[\xi_1^\wedge, \xi_2^\wedge]$ for $\xi_i = (v_{ix}, v_{iy}, \omega_i)^T$. Extract the result as a vector.
        \item Derive by hand the $3 \times 3$ matrix $\ad_\xi$.

        You should get: $\ad_\xi = \begin{bmatrix} \omega^\wedge & \bm{v}^\odot \\ \bm{0}^T & 0 \end{bmatrix}$ where $\bm{v}^\odot = \begin{bmatrix} v_y \\ -v_x \end{bmatrix} = -J\bm{v}$.
    \end{itemize}

\end{enumerate}

%==============================================================================
\section*{Problem 4: SE(2) Kinematics Simulation (15 points)}
%==============================================================================

The kinematic equation $\dot{X} = X \xi^\wedge$ describes rigid body motion on $\SE(2)$.

\begin{enumerate}[label=(\alph*)]
    \item (5 pts) \textbf{Unicycle Model:} A unicycle has forward velocity $v$ and yaw rate $\omega$. In the body frame, there is no lateral motion ($v_y = 0$), so $v_x = v$.
    \begin{itemize}
        \item Write the body-frame twist $\xi = (v_x, v_y, \omega)^T$ for the unicycle.
        \item Expand $\dot{X} = X \xi^\wedge$ to derive the ODEs: $\dot{\theta} = ?$, $\dot{x} = ?$, $\dot{y} = ?$
        \item For constant $v > 0$ and $\omega > 0$: What shape is the trajectory? What is the radius?
    \end{itemize}

    \item (10 pts) \textbf{Simulation:} Implement two integrators:
    \begin{itemize}
        \item \textbf{Euler:} $X_{k+1} = X_k (I + \Delta t \, \xi_k^\wedge)$
        \item \textbf{Lie group:} $X_{k+1} = X_k \Exp((\Delta t \, \xi_k)^\wedge)$
    \end{itemize}
    Simulate with $v = 1$ m/s, $\omega = 0.5$ rad/s, $\Delta t = 0.1$ s for 20 seconds. Start at $X_0 = I$.
    \begin{itemize}
        \item \textbf{Plot 1:} $(x, y)$ trajectories for both methods on the same figure.
        \item \textbf{Plot 2:} $\|R^T R - I\|_F$ vs time. Which method preserves $R^T R = I$? Why?
        \item \textbf{Exact solution:} For constant twist $\xi$, the exact solution is $X(t) = \Exp((t\xi)^\wedge)$. Verify that your Lie group integrator matches $\Exp((t\xi)^\wedge)$ to numerical precision at $t = 20$s. (This confirms the exponential map's meaning!)
    \end{itemize}
\end{enumerate}

%==============================================================================
\section*{Submission Checklist}
%==============================================================================

Submit a \textbf{single PDF to Gradescope} containing:
\begin{itemize}
    \item[$\square$] Problem 1: Normal subgroup concepts and examples
    \item[$\square$] Problem 2: Semi-direct product structure and normal subgroup proofs
    \item[$\square$] Problem 3: Exp/Log, $\Ad_X$, and $\ad_\xi$ derivations with code
    \item[$\square$] Problem 4: Kinematics derivation and simulation plots
\end{itemize}

\end{document}
